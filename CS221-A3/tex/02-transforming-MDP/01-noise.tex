\item \points{2a}
If we add noise to the transitions of an MDP, does the optimal value always get
worse? Specifically, consider an MDP with reward function $\text{Reward}
(s,a,s')$, states $\text{States}$, and transition function $T(s,a,s')$.  Let's
define a new MDP which is identical to the original, except that on each action,
with probability $\frac{1}{2}$, we randomly jump to one of the states that we
could have reached before with positive probability.

Formally, this modified transition function is:
\[T'(s,a,s')= \frac{1}{2} T(s,a,s') + \frac{1}{2} \cdot \frac{1}{\{ s'' : T(s, a, s'') > 0\}}.\]

Let $V_1$ be the optimal value function for the original MDP, and $V_2$ the
optimal value function for the modified MDP. Is it always the case that $V_1
(s_\text{start})\geq V_2(s_\text{start})$?  If so, prove it in |blackjack.pdf|
and put |return None| for each of the code blocks. Otherwise, construct a
counterexample by filling out |CounterexampleMDP| in |submission.py|.

% BEGIN_HIDE
% END_HIDE
