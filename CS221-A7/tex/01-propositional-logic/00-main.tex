\item {\bf Propositional Logic}

Write a propositional logic formula for each of the following English sentences
in the given function in |submission.py|. For example, if the sentence is
{\em ''If it is raining, it is wet''}, then you would write
|Implies(Atom('Rain'), Atom('Wet'))|, which would be $\text{Rain} \to
\text{Wet}$ in symbols (see |examples.py|).

{\em Note: Don't forget to return the constructed formula!}

\begin{enumerate}

  \item \points{1a}
{\em ``If it's summer and we're in California, then it doesn't rain.''}

  \item \points{1b}
{\em ``It's wet if and only if it is raining or the sprinklers are on.''}

  \item \points{1c}
{\em ``Either it's day or night (but not both).''}

\end{enumerate}

You can run the following command to test each formula:
\begin{lstlisting}
python grader.py 1a-0-basic
\end{lstlisting}

If your formula is wrong, then the grader will provide a counterexample, which
is a model that your formula and the correct formula don't agree on. For
example, if you accidentally wrote |And(Atom('Rain'), Atom('Wet'))| for
{\em ``If it is raining, it is wet''}, then the grader would output the
following:
\begin{lstlisting}
Your formula (And(Rain,Wet)) says the following model is FALSE, but it should be TRUE:
* Rain = False
* Wet = True
* (other atoms if any) = False
\end{lstlisting}

In this model, it is not raining and it is wet, which satisfies the correct
formula $\text{Rain} \to \text{Wet}$ (|TRUE|), but does not satisfy the
incorrect formula $\text{Rain} \wedge \text{Wet}$ (|FALSE|). Use these
counterexamples to guide you in the rest of the assignment.
