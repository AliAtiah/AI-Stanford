\item \points{1b}

Now fill out the |MinimaxAgent| class in |submission.py| using the above
recurrence. Remember that your minimax agent should work with any number of
ghosts, and your minimax tree should have multiple min layers (one for each
ghost) for every max layer.

Your code should also expand the game tree to an arbitrary depth. Score the
leaves of your minimax tree with the supplied |self.evaluationFunction|, which
defaults to |scoreEvaluationFunction|. The class |MinimaxAgent| extends
|MultiAgentSearchAgent|, which gives access to |self.depth| and
|self.evaluationFunction|. Make sure your minimax code makes reference to these
two variables where appropriate, as these variables are populated from the
command line options.

{\bf {\em Implementation Hints}}
\begin{itemize}
  \item {\bf Read the comments in submission.py thoroughly before starting to
  code!}

  \item Pac-Man is always agent 0, and the agents move in order of increasing
  agent index. Use |self.index| in your minimax implementation to refer to the
  Pac-Man's index. Notice that only Pac-Man will actually be running your
  |MinimaxAgent|.

  \item All states in minimax should be |GameState|s, either passed in to
  |getAction| or generated via |GameState.generateSuccessor|. In this
  assignment, you will not be abstracting to simplified states.

  \item You might find the functions described in the comments to the
  |ReflexAgent| and |MinimaxAgent| useful.

  % <!-- \item You might find the function
  % |GameState.getLegalActions| useful, which returns all the possible legal
  % moves, where each move is |Directions.X| for some X in the set
  % {|NORTH, SOUTH, WEST, EAST, STOP|}. As suggested before, also go through
  % the |ReflexAgent| code for descriptions of important methods, like
  % |GameState.getPacmanState(), GameState.getGhostStates(), GameState.getScore()|,
  % and so on. Other important methods are further documented inside
  % the |MinimaxAgent| class.</p> -->

  \item The evaluation function for this part is already written
  (|self.evaluationFunction|), and you should call this function without
  changing it. Use |self.evaluationFunction| in your definition of
  $V_\text{minmax}$ wherever you used $\text{Eval}(s)$ in part $1a$. Recognize
  that now we're evaluating {\em states} rather than actions. Look-ahead agents
  evaluate {\em future states} whereas reflex agents evaluate {\em actions} from
  the current state.

  \item If there is a tie between multiple actions for the best move, you may
  break the tie however you see fit.

  \item The minimax values of the initial state in the |minimaxClassic| layout
  are 9, 8, 7, -492 for depths 1, 2, 3 and 4 respectively. {\bf You can use
  these numbers to verify whether your implementation is correct.} Note that
  your minimax agent will often win, despite the dire prediction of depth 4
  minimax search, whose command is shown below. Our agent wins 50-70\% of the
  time: Be sure to test on a large number of games using the |-n| and |-q|
  flags.

  \begin{lstlisting}
python pacman.py -p MinimaxAgent -l minimaxClassic -a depth=4
\end{lstlisting}
\end{itemize}
{\em {\bf Further Observations}}

These questions and observations are here for you to ponder upon; no need to
include in the write-up.

\begin{itemize}
  \item On larger boards such as |openClassic| and |mediumClassic| (the
  default), you'll find Pac-Man to be good at not dying, but quite bad at
  winning. He'll often thrash around without making progress. He might even
  thrash around right next to a dot without eating it. Don't worry if you see
  this behavior. Why does Pac-Man thrash around right next to a dot?

  \item Consider the following run:

  \begin{lstlisting}
python pacman.py -p MinimaxAgent -l trappedClassic -a depth=3
\end{lstlisting}

  Why do you think Pac-Man rushes the closest ghost in minimax search on
  |trappedClassic|?

  % <!--
  % \item To increase the search depth achievable by your agent, remove the |Directions.STOP| action from Pac-Man's list of possible actions. Depth 2 should be pretty quick, but depth 3 or 4 will be slow. Don't worry, the next problem will speed up the search somewhat.

  % \item You are welcome to, but not required to, remove |Directions.STOP| as a valid action (this is true for any part of the assignment).
  % -->

\end{itemize}